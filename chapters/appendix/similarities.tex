\newpage

\section{Similarities}
\begin{prob}
Based on your experience with the stretching activity, write a definition of dilation.  Be sure to indicate (1) what it takes to specify the transformation, and (2) how to produce the image of a given point.  
\vspace{0.6in}
\end{prob}

\begin{prob}
Based on your experience with the stretching activity, describe for a dilation: 
\begin{enumerate}
\item What happens to line segments? 
\vspace{0.2in}
\item What happens to angles?  
\vspace{0.2in}
\item What happens to lines passing through the center of the dilation?
\vspace{0.2in}
\item What happens to lines not passing through the center of the dilation?
\vspace{0.2in}
\end{enumerate}
\end{prob}

%Every dilation has the following properties:
%\begin{enumerate}[(i)]
%\item It maps lines to lines, rays to rays, and segments to segments.
%\item It changes distance by a factor of $r$, where $r$ is the scale factor of the dilation.
%\item It maps every line passing through the center of dilation to itself, and it maps every line not passing through the center of the dilation to a parallel line.  
%\item It preserves angle measure.
%\end{enumerate}

\begin{definition}
A geometric figure is \emph{simliar} to another if the second can be obtained from the first by a sequence of rotations, reflections, translations, and dilations.  
\end{definition}

\begin{prob}
For each of the pairs of objects on the following pages, do the following:  
\begin{enumerate}
\item Trace the smaller figure on plastic.  Then close one eye and try to hold the plastic between your eye and the paper so that the tracing ``exactly'' covers the larger figure.   Be sure that the plane of the paper and the plane of the plastic are parallel.  (Why does this matter?) 
\item If the objects are similar, find a sequence of rotations, reflections, translations, and dilations that takes one figure onto the other.  
\item If the objects are similar, try to find a single dilation that demonstrates the similarity.   If you cannot find such a dilation, explain how you know you cannot.  
\end{enumerate}
\end{prob}

\includegraphics{../graphics/similarTriangles1}
\includegraphics{../graphics/similarTriangles2}

\includegraphics{../graphics/similarTriangles3}
\includegraphics{../graphics/similarTriangles4}

\newpage

$$\includegraphics{../graphics/similarCircles1}$$
\begin{prob}
Describe a general (and foolproof) way of demonstrating that any two circles are similar.\standardhs{G-C.1} 
\end{prob}

\[
\includegraphics{../graphics/similarParabolas1}
\includegraphics{../graphics/similarParabolas2}
\]

\begin{prob}
Describe a general (and foolproof) way of demonstrating that any two parabolas are similar. 
\end{prob}



