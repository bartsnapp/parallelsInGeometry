\newpage
\section{The Euler Line} 



\begin{prob} 
Use \textsl{GeoGebra} to construct the circumcenter of a
triangle. Hide all extraneous lines and points. Label this point $C$.
\end{prob}

\begin{prob} 
Use \textsl{GeoGebra} to construct the centroid of the same triangle. Hide
all extraneous lines and points. Label this point $N$.
\end{prob}
\begin{prob} 
Use \textsl{GeoGebra} to construct the orthocenter of the same triangle. Hide
all extraneous lines and points. Label this point $O$.
\end{prob}

\begin{prob} 
Connect $C$ and $O$ with a segment. Did a miracle happen---or has
\textsl{GeoGebra} been a naughty monkey?
\end{prob}

%% Round two---let's talk about circles baby!
%% \begin{prob} 
%% Keeping the same triangle as used in the previous problems, use
%% \textsl{GeoGebra} to mark the midpoint of the segment that connects
%% $C$ and $O$. Label this point $M$.
%% \end{prob}


%% \begin{prob} 
%% On the same triangle use \textsl{GeoGebra} to mark the midpoints of
%% each side. Note, you should just be able to ``unhide'' them as they
%% are already there.
%% \end{prob}

%% \begin{prob} 
%% On the same triangle use \textsl{GeoGebra} to mark where the altitudes
%% meet the lines containing the sides of the triangle. Hide all
%% extraneous lines and points.
%% \end{prob}

%% \begin{prob}
%% On the same triangle use \textsl{GeoGebra} to mark the midpoints of
%% the segments joining the orthocenter and the vertexes. Hide all
%% extraneous lines and points.
%% \end{prob}


%% \begin{prob}
%% On the same triangle use \textsl{GeoGebra} to draw a circle centered
%% at $M$ that goes through one of the midpoints of the triangle.
%% \end{prob}

%% \begin{prob} 
%% Did a miracle happen---or has \textsl{GeoGebra} morphed into
%% \textsl{Geogezilla}?
%% \end{prob}

