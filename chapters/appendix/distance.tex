\newpage

\section{Distance Across Geometries}

\begin{prob}
Describe the meaning of the absolute value of $x$ in at least three different ways.  
\end{prob}

\fixnote{Replace this activity with newer version.}

% the magnitude of x
% the distance from 0
% |x| = \brace x, if x \ge 0; -x, if x < 0
% Some students will say ``always positive.''  This is correct, but insufficient.  

\begin{prob}
Use a meaning of absolute value to explain the solution(s) to the equation $|x - 5| = 8$.  Be sure to indicate how you know you have all the solutions.  
\end{prob}


\begin{prob} 
Let $A = (x_1,y_1)$ and $B = (x_2,y_2)$.  
\begin{enumerate}
\item Write a formula for the Taxicab distance $d_T(A,B)$.
\item Compare to the Euclidean distance $d_E(A,B)$.
\end{enumerate}
\end{prob}

\begin{prob}Consider circles in city geometry and Euclidean geometry.  
\begin{enumerate}
\item Write a definition of ``circle.''
\item Use your definition of circle to write an equation for a city geometry circle of radius 1 centered at the origin. 
\item Draw a graph of your city geometry circle. 
\item Explain how you know your graph is correct.  (Hint:  Consider separate cases.)  
\item Compare to the equation of a Euclidean circle.  
\end{enumerate}
\end{prob}

\begin{prob}
In city geometry, write the equation of a circle of radius $r$, centered at $(u, v)$.  Compare to Euclidean geometry. 
\end{prob}


