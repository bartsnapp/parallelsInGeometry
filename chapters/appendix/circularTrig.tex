\newpage 

\section{Circular Trigonometry}
As we have seen, right triangle trigonometry is restricted to acute angles.  But angles are often obtuse, so it is quite useful to extend trigonometry to angles greater than $90^\circ$.  Here is one approach:  Place the angle with the vertex at the origin in the coordinate and with one side of the angle (the initial side) along the positive $x$-axis.  Measure to the other side of the angle (the terminal side) as a counter-clockwise rotation about the origin.   

%\begin{prob}
%Now we can find the area of a triangle given two sides and an angle.\standardhs{G-SRT.9}  
%\end{prob}
%
%% Laws of Sines and Cosines \standardhs{G-SRT.10}, \standardhs{G-SRT.11}

Because angles are often about rotation, angles greater than $180^\circ$ can make sense, too.  And negative angles can describe rotation in the opposite direction.  If we consider the angle to change continuously, then rotation about the origin creates a situation that repeats every $360^\circ$.  This repetition provides the foundation for modeling lots of repetitive (periodic) contexts in the real world.  For this modeling, we need \emph{circular trigonometry}, which turns out to be much cleaner if (1) angles are measured not in degrees but in a more ``natural'' unit, called radians; and (2) we use \emph{the unit circle}, which is a circle of radius 1 centered at the origin.   The steps are as follows:  

\begin{enumerate}
\item Understanding radian measure.\standardhs{G-C.5}, \standardhs{F-TF.1}

\item Using the unit circle to extend trigonometry to angles of any measure.\standardhs{F-TF.2}

\item Choosing and using trig functions to model periodic phenomena.\standardhs{F-TF.5}

\end{enumerate}

\fixnote{Complete this activity or section.}
% Fluency finding trig functions of special angles in radian measure.\standardhs{F-TF.3}

