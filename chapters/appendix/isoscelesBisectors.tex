\newpage

\section{Isosceles Bisectors}

\begin{theorem}[Isosceles Triangle Theorem]
If two sides of a triangle are congruent, then the angles opposite those sides are congruent. 
\end{theorem}

\begin{prob}
Prove the Isosceles Triangle Theorem.  (Hint: In your explorations, you have noticed that in most triangles the median, perpendicular bisector, angle bisector, and altitude to a side lie on four different lines.  So if you draw a new line in your diagram, be sure to indicate which of these lines you are drawing.)
\end{prob}

\begin{prob}
Prove the Isosceles Triangle Theorem without drawing another line.  Hint:  Is there a way in which the triangle is congruent to itself? 
\end{prob}

\begin{prob}
State the converse of the Isosceles Triangle Theorem and prove it.  
\end{prob}

\begin{prob}
Prove that the points on the perpendicular bisector of a segment are \emph{exactly those} that are equidistant from the endpoints of the segment.  Note that the phrase \emph{exactly those} requires that we prove a simpler statement as well as its converse:   
\begin{enumerate}
\item Prove that a point on the perpendicular bisector of a segment is equidistant from the endpoints of that segment.
\item Prove that a point that is equidistant from the endpoints of a segment lies on the perpendicular bisector of that segment.
\end{enumerate}
\end{prob}

\begin{prob}
Prove that the perpendicular bisectors of a triangle are concurrent.  Hint:  Name the intersection of two of the perpendicular bisectors and then show that it must also lie on the other two.  (This is a standard approach for showing the concurrency of three lines.)  
\end{prob}

\begin{prob}
Draw a line and a point not on the line.  Describe how to find the \emph{exact} distance from the point to the line. 
\end{prob}

\begin{prob}
Prove that the points on an angle bisector are \emph{exactly those} that are equidistant from the sides of the angle. 
\end{prob}

\begin{prob}
Prove that the angle bisectors of a triangle are concurrent. 
\end{prob}

