\newpage

\section{Quadrilateral Diagonals}

Imagine you are working at a kite factory and you have been asked to design a new kite.  The kite will be a quadrilateral made of synthetic cloth, and it will be formed by two intersecting rods that serve as the diagonals of the quadrilateral and provide structure for the kite.  

\begin{prob}
To get started, review the definitions of all special quadrilaterals.  Be sure to include \emph{kite} on your list.  
\end{prob}

\begin{prob}
To consider the possible kite shapes, your first task is to describe how conditions on the diagonals determine the quadrilateral.  Use spaghetti to model the intersecting rods, and use paper and pencil to draw the rod configurations and resulting kite shapes.   Explore diagonals of various lengths, of the same length, and of different lengths.  Explore various places at which to attach the diagonals to each other, including at one or both of their midpoints.  Explore various angles that the diagonals might make with each other at their intersection, including the possibility of being perpendicular.  
\end{prob}

\begin{prob}
Summarize your findings in a table organized like the one below.  

\renewcommand\arraystretch{2}
\renewcommand\tabcolsep{6pt}
\begin{table}[h]
\begin{tabular}{|l|l|r|l|}
\hline
\multicolumn{1}{|c|}{\begin{tabular}[c]{@{}c@{}}Diagonal\\ Conditions\end{tabular}} & Quadrilateral & Definition & Other Key Properties \\ \hline
                                                                                    &               &            &                  \\ \hline
                                                                                    &               &            &                  \\ \hline
                                                                                    &               &            &                  \\ \hline
                                                                                    &               &            &                  \\ \hline
                                                                                    &               &            &                  \\ \hline
\end{tabular}
\end{table}

\end{prob}




