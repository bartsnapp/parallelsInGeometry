
\newpage

\section{Parametric Plots of Circles}

In this activity we'll investigate parametric plots of circles.



\begin{prob} 
One problem with the standard form for a circle, even the form for the unit circle
\[
x^2 + y^2 = 1,
\]
is that it is somewhat difficult to find points on the circle. We
claim that for any value of $t$,
\begin{align*}
x(t) &= \cos(t)\\
y(t) &= \sin(t) 
\end{align*}
will be a point on the unit circle. Can you give me some explanation
as to why this is true? Two hints, for two answers: The unit circle;
The Pythagorean identity.
\end{prob} 

\begin{prob}
Another way to think about parametric formulas for circles is to imagine 
\begin{align*}
x(\theta) &= \cos(\theta)\\
y(\theta) &= \sin(\theta) 
\end{align*}
where $\theta$ is an angle. What is the connection between value of
$\theta$ and the point $(x(\theta), y(\theta))$?
\end{prob}

\begin{prob}
One way to think about parametric formulas for circles is to imagine 
\begin{align*}
x(t) &= \cos(t)\\
y(t) &= \sin(t) 
\end{align*}
as ``drawing'' the circle as $t$ changes. Starting with $t=0$,
describe how the circle is ``drawn.''  Make a table of values of $t$, $x$, and $y$.  Use values of $t$ that are special angles.  Includes values of $t$ that are negative as well as some values of $t$ that are greater than $2\pi$.  
\end{prob}

\begin{prob}
One day you accidentally write down
\begin{align*}
x(t) &= \sin(t)\\
y(t) &= \cos(t) 
\end{align*}
Again, make a table of values of $t$, $x$, and $y$
What happens now? Do you still get a circle? How is this different
from what we did in the previous question?
\end{prob}

\begin{prob}
Do the formulas 
\begin{align*}
x(t) &= \cos(t)\\
y(t) &= \sin(t) 
\end{align*}
define a function? Discuss. Clearly identify the
domain and range as part of your discussion.  Remember, the domain is the set of input values and the range is the set of output values.  
\end{prob}

\begin{prob}
Reason with your previous tables of $x$- and $y$-values to determine the graph of the following parametric equations. 
\begin{align*}
x(t) &= 2\cos(t) + 3\\
y(t) &= 2\sin(t) - 4 
\end{align*}
Explain your reasoning.  
\end{prob}

\begin{prob} 
Now we will go backwards.  The standard form for a circle centered at a point $(a,b)$ with radius $c$ is given by
\[
(x-a)^2 + (y-b)^2 = r^2.
\]
Explain why this makes perfect sense from the definition of a circle. 
\end{prob} 


\begin{prob}
Here are three circles
\[
(x-1)^2 + (y+2)^2 = 4^2 \qquad (x+4)^2 + (y-2)^2 = 8 \qquad x^2+y^2 -4x+6y= 12.
\]
Convert each of these circles to parametric form.
\end{prob}

