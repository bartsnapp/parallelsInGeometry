\newpage

\section{Parallels}
In the following problems, you may assume the following: 

\begin{postulate}[Parallel Postulate]
Given a line and a point not on the line, there is exactly one line passing through the point which is parallel to the given line.
\end{postulate}

You may also use previously-established results, such as the following: 
\begin{itemize}
\itemsep -3pt
\item The measures of adjacent angles add as they should.
\item A straight angle measures $180^\circ$.  
\item A $180^\circ$ rotation about a point on a line takes the line to itself.  
\item A $180^\circ$ rotation about a point off a line takes the line to a parallel line.  
\end{itemize}

Now you may get started! 

\begin{prob}
Use adjacent angles to prove that vertical angles are equal.    
\end{prob}

\begin{prob}
Now use rotations to prove that vertical angles are equal.
\end{prob}

\begin{prob}
Prove:  Alternate interior angles and corresponding angles of a transversal with respect to a pair of parallel lines are equal.
\end{prob}

\begin{prob}
Prove: If a pair of alternate interior angles or a pair of corresponding angles of a transversal with respect to two lines are equal, then the lines are parallel.
\end{prob}

\begin{prob}
The previous two problems seem almost identical to one another.  How are they different?  
\end{prob}

\begin{prob}
Prove:  The angle sum of a triangle is $180^\circ$.
\end{prob}

