
\newpage

\section{Parametric Equations}

\begin{prob} How do you plot $y=3x-1$ with a table?  Explain what the inputs and outputs are.  Exactly what do you plot.  
\end{prob}

\begin{prob} How do you plot $y = \frac{1}{x}$.  Where can you connect the dots?  Where do you need to be careful and why?  
\end{prob}

When graphs are given by \emph{parametric equations}, the coordinates $x$ and $y$ may be given as functions of $t$, often thought of as "time."  To begin graphing parametric equations, make a table of values for $t$, $x$, and $y$, and then plot the order pairs $(x, y)$.  

\begin{prob}
 Consider the parametric equation $(x,y) = (2t+3,-t-4)$.  
\begin{enumerate}
\item Graph the equation.  It might help to note various values of $t$ on your graph. 
\item Describe the graph and explain why it looks the way it does.  
\item Locate the points corresponding to $t=\frac{2}{3}$, $\frac{5}{4}$, $3.1$, and $\pi$.  
\item Why is it okay to connect the dots?  Consider what happens to the $x$ and $y$ coordinates near and between points you have already plotted.  
\item Where are the input and output values in the graph of a parametric equation?  
\end{enumerate}

\end{prob}
\begin{prob} Graph the equation $(x,y) = (2t+5,-t+1)$. How does it compare with your previous graph?  Explain. 
\end{prob}
\begin{prob} Graph the equations $(x,y) = (3t+1, 4)$ and $(x,y) = (-1, 2+t)$.  Explain why the graphs look as they do.  
\end{prob}

A \emph{vector} has both direction and magnitude (i.e., length).  In this course, vectors will often be given as ordered pairs, and they may be drawn or imagined as arrows from the origin to the given point.  For example, the vector $(3,2)$ can be represented as an arrow from $(0,0)$ to $(3,2)$.  But the position of the vector is unimportant.  

\begin{prob}
Explain why an arrow from $(1,6)$ to $(4,8)$ describes the vector $(3,2)$.  
\end{prob}

\begin{prob}
What vector may be represented by an arrow from $(6,4)$ to $(2,1)$?  
\end{prob}

\begin{prob} Consider the equation $(x,y)=(2,1)+t(-1,3)$.  
\begin{enumerate} 
\item Graph the equation.  
\item Use the ideas of a starting point and a direction vector to explain why the graph looks the way it does.   
\item Pick an arbitrary point on your graph and describe how to arrive at that point using the starting point and scaling the direction vector.  
\end{enumerate}
\end{prob}
\begin{prob}  Graph the equation $(x,y) = (2,1) + t(2,-6)$.  Compare and contrast this problem with the previous problem.  
\end{prob}
\begin{prob}  Write a parametric equation for the line containing $(-3, 2)$ and $(2, 1)$.  
\end{prob}
\begin{prob}  Write a parametric equation for the line containing $(1, 2)$ and $(2, -1)$.

\end{prob}
\begin{prob} Write a parametric equation for the line containing the points $(a,b)$ and $(c,d)$.  

\end{prob}
\begin{prob}  Consider the line containing the points $A=(2, 4)$ and $B=(-1, 8)$.
\begin{enumerate}
\item Find the coordinates of the point 2/3 of the way from $A$ to $B$.  
\item Find the coordinates of the point 3/5 of the way from $A$ to $B$.
\item Find the coordinates of the point 5/4 of the way from $A$ to $B$.  
Find the coordinates of the point $p/q$ of the way from $A$ to $B$.  
\item What would it mean for $p/q$ to be greater than 1?  Explain
\item What would it mean for $p/q$ to be negative?  Explain.  
\item What geometric object will result if $p/q$ varies through all possible rational numbers?  Explain.   
\item  Find the coordinates of the point $p/q$ of the way between $(a, b)$ and $(c, d)$.  
\end{enumerate}

\end{prob}
\begin{prob}  What are all the possible slopes of lines in the plane?  Explain.  

\end{prob}
\begin{prob}  Pick your favorite rational number $\frac{a}{b}$.  Make a table of pairs $(x, y)$ such that $\frac{y}{x}=\frac{a}{b}$, and plot the ordered pairs.  Be sure to include some negative values.  Draw the graph of all such ordered pairs, including all possible real values for $x$ and $y$.  What do you notice about your graph and your table?  What it would mean to say that your graph represents the rational number $\frac{a}{b}$?  Does your graph include $(0, 0)$?  What does that say about $\frac{0}{0}$?  

\end{prob}

