\newpage
\section{Centers, Circles, and Lines Oh My!}

In this activity, we are going to explore the basic centers, circles,
and lines related to triangles.  \fixnote{Use Geogebra to explore and construct.  Use new sketches for each.}

\begin{prob} Here are some easy questions to get the brain-juices flowing!
\begin{enumerate} 
\item Place two points randomly in the plane. Do you expect to be able to
draw a single line that connects them?
\item Place three points randomly in the plane. Do you expect to be able to
draw a single line that connects them?
\item Place two lines randomly in the plane. How many points do you expect
them to share?
\item Place three lines randomly in the plane. How many points do you expect
all three lines to share?
\item Place three points randomly in the plane. Will you (almost!) always be
able to draw a circle containing these points? If no, why not? If yes,
how do you know?
\item Place four points randomly in the plane. Do you expect to be able to
draw a circle containing all four at once? Explain your reasoning.
\end{enumerate}
\end{prob}


Now, to make things more pleasant, I suggest you let \textsl{GeoGebra}
help you out.

\begin{prob} 
Draw yourself a triangle. Now construct the perpendicular bisectors of
the sides---notice anything? Does this work for every triangle?
\end{prob}

\begin{prob}
Now bisect the angles---notice anything? Does this work for every
triangle?
\end{prob}

\begin{prob}
Now construct the lines containing the altitudes---notice anything?
Does this work for every triangle?
\end{prob}

\begin{prob}
Now construct the medians---notice anything? Does this work for every
triangle?
\end{prob}

\begin{prob}
Now construct the circumcircle using the ``three-point-circle'' tool.
Construct the circumcenter using the ``midpoint-or-center''
tool. Notice anything in connection to the lines drawn above? Does
this work for every triangle?
\end{prob}

\begin{prob}
The incenter is found via the intersection of the angle
bisectors. Construct the incircle.
\end{prob}

\begin{prob}
Fill in the following handy chart summarizing the information you found
above. 
\[
\begin{tabular}{| l || c | c | c | c |}
\hline
  & \begin{minipage}{15ex}\minipad  what are they? (draw pictures) \minipad\end{minipage}& \begin{minipage}{10ex}\minipad associated point? \minipad\end{minipage} & \begin{minipage}{10ex}\minipad always inside the triangle? \minipad\end{minipage} & meaning? \\ \hline\hline 
\begin{minipage}{15ex}\minipad perpendicular \\ bisectors \minipad\end{minipage} & \rule[0mm]{0mm}{7mm}\hspace{20mm}  &\hspace{20mm}  &\hspace{20mm}  & \hspace{20mm} \\ \hline
\begin{minipage}{7ex}\minipad angle \\ bisectors \minipad\end{minipage} & \rule[0mm]{0mm}{7mm}   &  &  & \\ \hline
\begin{minipage}{12ex}\minipad lines \\ containing altitudes \minipad\end{minipage} & \rule[0mm]{0mm}{7mm}   &  &  &  \\ \hline
\begin{minipage}{12ex}\minipad lines \\ containing the medians  \minipad\end{minipage} & \rule[0mm]{0mm}{7mm}   &  &  &   \\ \hline
\end{tabular}
\]
Be sure to put this in a safe place like in a safe, or under your bed.
\end{prob}
