\newpage

\section{Supplemental Problems}

\subsection{Parametric Equations}


\begin{prob} 
Consider a nonzero vector defined by the ordered pair $(a,b)$. If $\|
(a,b)\|$ is the magnitude of this vector, \textbf{use algebra} to
explain why
\[
\frac{(a,b)}{\|(a,b)\|}
\]
is a new vector whose magnitude is $1$ and whose direction is the same
as $(a,b)$.
\end{prob} 

\begin{prob}
Suppose you have a parametric plot defined by $x(t)$ and $y(t)$.
\begin{enumerate}
\item Compare and contrast the plots of
\[
\bigg(x(t),y(t)\bigg)\qquad\text{and}\qquad\bigg(x(t-6),y(t-6)\bigg).
\]
\item Suppose that there are two bugs whose positions are given by:
\[
\mathrm{bug}_1(t) = \bigg(x(t),y(t)\bigg)\qquad\text{and}\qquad\mathrm{bug}_2=\bigg(x(t-6),y(t-6)\bigg).
\]
where $t$ represents time in seconds. Describe what happens as $t$
runs from $0$ seconds to $36$ seconds.

\item Now suppose that there are two bugs whose positions are given
  by:
\[
\mathrm{bug}_1(t) = \bigg(x(t),y(t)\bigg)\qquad\text{and}\qquad\mathrm{bug}_2=\bigg(x(t)-6,y(t)-6\bigg).
\]
where $t$ represents time in seconds. Describe what happens as $t$
runs from $0$ seconds to $36$ seconds.
\end{enumerate}
\end{prob} 

\begin{prob}
Find the intersection of the lines
\begin{align*}
x_1(t) &= -6 + 9t & x_2(t) &= 3+t \\
y_1(t) &= 3-2t &  y_2(t) &= -4-2t 
\end{align*}
If $(x_1(t),y_1(t))$ gives the position of $\mathrm{jogger}_1$ and
$(x_2(t),y_2(t))$ gives the position of $\mathrm{jogger}_2$, what is
the significance of the point of intersection of these lines, from the
perspective of the joggers?
\end{prob}

\begin{prob}
A bug moves according to the following parametric equations, where t is measured in seconds and $x$ and $y$ are measured in centimeters:  $x = 2t^2$, $y = t-2$.  (Suppose $t$ can be any real number.)   
\begin{enumerate}
\item Describe the path of the bug.  
\item Is the bug's position a function of time?  
\item On the path, is $y$ a function of $x$?  
\item Is $x$ a function of $y$?  
\item If you know one of $x$, $y$, or $t$, can you determine the other two?  How does this question relate to the previous two questions?  
\item In school mathematics, students are often given a graph and asked, ``Is it a function.''  Explain why this is a poor question.  What better questions could you ask?  
\end{enumerate}
\end{prob}

\subsection{Absolute Value, Distance, and City Geometry}
\begin{prob}
Consider the following equations:  
\setlength{\arraycolsep}{12pt}
\setlength{\extrarowheight}{3pt}
\[
\begin{array}{cccc}
x^2-y^2=0    &   x^2=y^2   &   |y|=|x|   &   y= \pm x \\
(x-y)(x+y)=0  &   x= \pm y   &   y = \pm|x|   & x = \pm|y|
\end{array}
\]
\begin{enumerate}
\item Which equations are equivalent to which other equations?  Say how you know.  (Be sure to state what it means for the equations to be equivalent.)
\item For each set of equivalent equations, graph the solution set, and describe how each of the equations provides a different way about thinking about that solution set.  
\end{enumerate}
\end{prob}

\begin{prob}
Distance formulas and circle equations across dimensions.  
\begin{enumerate}
\item What is the (Euclidean) distance formula in 2 dimensions, on the $xy$-plane?  
\item What is the distance formula in 3 dimensions?
\item What is the distance formula in 1 dimension?
\item Write an equation of the circle of radius $r$ and center $(a, b)$.  
\item Explain how a circle is a one-dimensional figure living in a two-dimensional ``space.''
\item In three-dimensional space, write an equation of the two-dimensional ``circle'' of radius $r$ and center $(a, b, c)$.  
\item In one-dimensional space, write an equation of the zero-dimensional ``circle'' of radius $r$ and center $a$.  
\end{enumerate}
\end{prob}

%\begin{prob}
%Distances across dimensions.  
%\begin{enumerate}
%\item Find all numbers that are equidistant from 3 and 8.  
%\item Find all points $(x, y)$ that are equidistant from $(3, 0)$ and $(8, 0)$.  
%\item Find all points $(x, y, z)$ that are equidistant from $(3, 0, 0)$ and $(8, 0, 0)$.  
%\item Find all real numbers that are twice as far from 3 as they are from 8.
%\item Find all points $(x, y)$ that are twice as far from $(3, 0)$ as they are from $(8, 0)$.
%\item Find all points $(x, y, z)$ that are twice as far from $(3, 0, 0)$ as from $(8, 0, 0)$.
%\end{enumerate}
%\end{prob}
%
%4.5.  One way of solving problems 4a and 4d is as follows: 
%(1)	Use a distance formula from problem 3 to express the distances from x to 3 and from x to 8; 
%(2)	Use an equation to relate the distances from part (1); and 
%(3)	Thinking of the two sides of the equation as functions, graph the two functions and look for intersections of the graphs. 
%How can you use this approach to predict how many solutions you will find?

\begin{prob} 
Recall the method in Euclidean geometry of constructing an equilateral triangle on a given segment.  Suppose a ``city geometry compass'' draws a city geometry circle.  Imagine using such a ``city geometry compass'' below.  
\begin{enumerate}
\item Construct a ``city geometry equilateral triangle'' on the segment defined by the
  points $(0,0)$ and $(4,0)$. Explain your steps.
\item Now construct a ``city geometry equilateral triangle'' on the segment defined by the
  points $(0,0)$ and $(2,2)$. Explain your steps.
\item Will the construction always give a (unique!) equilateral triangle? What does ``unique'' mean in this context? Give a detailed discussion.  
\end{enumerate}
\end{prob} 

%
%\begin{prob} 
%Consider a nonzero vector defined by the ordered pair $(a,b)$. If $\|
%(a,b)\|$ is the magnitude of this vector, \textbf{use algebra} to
%explain why
%\[
%\frac{(a,b)}{\|(a,b)\|}
%\]
%is a new vector whose magnitude is $1$ and whose direction is the same
%as $(a,b)$.
%\end{prob} 

\subsection{Measurement}
\begin{prob}
If the perimeter of a rectangle is 20 feet, what is the most one can say about the rectangle's area?  If the perimeter of any simple closed 2-dimensional shape is 20 feet, what is the most anyone can say about its area?
\end{prob}

\begin{prob}
If the surface area of a rectangular prism is 20 square feet, what is the most one can say about the prism's volume?  If the surface area of any simple closed 3-dimensional shape is 20 square feet, what is the most one can say about its volume? 
\end{prob}

\begin{prob}
Why do cute furry animals curl up to stay warm in the winter?  Why are most ugly desert reptiles long and skinny?
\end{prob}

\begin{prob}Simple closed curve A is contained entirely inside simple closed curve B.  
\begin{enumerate}
\item True or False:  The area enclosed by A is less than the area enclosed by B. Explain
\item True or False:  The perimeter of A is less than the perimeter of B. Explain.  
\end{enumerate}
\end{prob}

\begin{prob}
Is it correct to say that``area is length times width''?  Think about what these three quantities mean.  When would it be correct in the numerical sense and why?  (Make sure you use the meaning of multiplication.)   
\end{prob}

\begin{prob}
The apothem of a regular polygon is defined to be the shortest distance from the center of the polygon to an edge.
\begin{enumerate}
\item There is a nice relationship between the apothem, perimeter, and area for a regular polygon.  See if you can find it. (Hint:  Split the polygon into congruent triangles from its center and find the area of the polygon in terms of the apothem and perimeter.)  You can assume you know the area of a triangle $=\frac{1}{2}$(Length of Base)(Length of Height).
\item What does this result say about the area of a circle?  Explain. (Assume you know the circumference of a circle is $2\pi(radius)$.)
\end{enumerate}
\end{prob}

\begin{prob}
 Is it correct to say that ``volume is length times width times height''? What must be true about a figure so that the numerical volume can be more easily measured by ``area times height''?
\end{prob}

\begin{prob}
Are there figures for which there is no formula for measuring length, area, and volume?  Explain.  What does your answer to this question imply about the teaching of geometric measurement?
\end{prob}

\begin{prob}
Convert 25 yards to meters (and 25 meters to yards) using ``2.54 cm in each inch'' as the only Metric-English unit conversion.  Explain the conversion using the meanings of multiplication and division (In particular, when you divide, what kind of division is it?)
\end{prob}

\begin{prob}
Now convert 25 square yards to square meters and 25 square meters to square yards.  Do the same with cubic yards and cubic meters.
\end{prob}

\fixnote{Add problems about oriented area, planimeter.}  

\begin{prob}
TIMSS 1995 problem about changing a border for equal areas. 
\end{prob}

\begin{prob}
Using oriented area, given a quadrilateral $ABCD$, and any point $O$ in the plane, $$\text{area} ABCD = \text{area}\triangle OAB + \text{area}\triangle OBC + \text{area}\triangle OCD + \text{area}\triangle ODA.$$
\end{prob}
  
\begin{prob}
A parallelogram with vertices at $(0,0)$, $(a,b)$, $(c,d)$, and $(a+c,b+d)$ has (oriented) area $ad-bc$. 
% Draw a big rectangle around it, and subtract the stuff we don't want.
% Row reduction of a matrix as parallelograms with same base, same height.  Area (and determinant) is constant.  
\end{prob}

\begin{prob}
If $ad-bc = 0$, then $(a,b)$ and $(c,d)$ are parallel.  What about the converse?  
\end{prob}

