\newpage

\section{Parallels}
School geometry (or any geometry) is based on a set of assumptions, sometimes called axioms or postulates, which are merely statements that are accepted without proof.  We propose the following set of assumptions for school geometry:  
\begin{itemize}
\item[(A1)] Through two distinct points passes a unique line.
\item[(A2)] Given a line and a point not on the line, there is exactly one line passing through the point which is parallel to the given line (Parallel postulate).
\item[(A3)] The points on a line can be placed in one-to-one correspondence with the real numbers so that differences measure distances (Ruler postulate).  
\item[(A4)] The rays with a common endpoint can be numbered so that differences measure angles and so that straight angles measure $180^\circ$ (Protactor postulate). 
\item[(A5)] Every basic rigid motion (rotation, reflection, or translation) has the following properties:
\begin{enumerate}[(i)]
\item It maps a line to a line, a ray to a ray, and a segment to a segment.
\item It preserves distance and angle measure.
\end{enumerate}
\item [(A6)] Areas of geometric figures have the following properties: 
\begin{enumerate}[(i)]
\item Congruent figures enclose equal areas.
\item Area is additive, i.e., the area of the union of two regions that overlap only at their boundaries is the sum of their areas. 
%\item Area is measured by tiling a region with a two-dimensional unit (such as a square) and parts of the unit, without gaps or overlaps. 
\item A rectangle with side-lengths $a$ and $b$ has area $ab$, where $a$ and $b$ can be any real numbers.
\end{enumerate}
%\item[(A7)] Every dilation has the following properties:
%\begin{enumerate}[(i)]
%\item It maps lines to lines, rays to rays, and segments to segments.
%\item It changes distance by a factor of $r$, where $r$ is the scale factor of the dilation.
%\item It maps every line passing through the center of dilation to itself, and it maps every line not passing through the center of the dilation to a parallel line.  
%\item It preserves angle measure.
%\end{enumerate}
\end{itemize}

Note:  In addition to these geometric assumptions, we assume the properties of the algebra of real numbers.  
%In a formal treatment of the algebra of real numbers, some of these would be axioms, others 
%would be definitions, and still others would be theorems, yet it is not necessary in school mathematics 
%to distinguish among them.  
Some of these properties (e.g., the distributive property of multiplication over addition) have well-known names.  But it is neither necessary nor instructive to ensure that every such property have a name.  

These assumptions can be remembered easily in the following chunks:  
\begin{enumerate}
\item (A1) and (A2):  Points, lines, and parallel lines behave as they should. 
\item (A3) and (A4):  Distance and angle measure behave as they should. 
\item (A5):  Basic rigid motions behave as they should.
\item (A6):  Area behaves as it should.  
\end{enumerate}

\begin{prob}
Prove:  Given two distinct lines, either they are parallel, or they have exactly one
point in common.  
\end{prob}

%Lemma 2. If three lines L1, L2, and L3 have the property that L1 || L2 and L2 || L3,
%then L1 || L3.

\begin{prob}
With the ruler postulate, we can provide a definition of ``betweenness.''  If points $A$, $X$, and $B$ are on a line $l$, we say that $X$ is \emph{between} $A$ and $B$ if $AX + XB = AB$. 
\begin{enumerate}
\item Use the concept of betweenness to define line segment $\overline{AB}$.  
\item Use the concept of betweenness to define ray $\overrightarrow{AB}$. 
\end{enumerate}
\end{prob}

\begin{prob}
Use the protractor postulate to provide a definition of adjacent angles, analogous to betweenness for distances.  
\end{prob}

\begin{prob}
Prove:  Let $l$ be a line and $O$ be a point on $l$. Let $R$ be the $180^\circ$
rotation around $O$. Then $R$ maps $l$ to to itself.  (Hint:  Pick points $P$ and $Q$ on $l$ so that $O$ is between them, and consider the straight angle $\angle POQ$.)
\end{prob}

\begin{prob}
Prove:  Let $l$ be a line and $O$ be a point \emph{not} lying on $l$. Let $R$ be the 180-degree
rotation around $O$. Then $R$ maps $l$ to a line parallel to itself.  (Hint: Suppose not.)
\end{prob}

%Proof.  Suppose R(L) and L have a point Q in
%common. Because Q is in R(L), there is a point P in L, so that R(P) = Q. Because
%R is a 180-degree rotation around O, the three points P, Q, and O lie in a line `. But
%Q is by assumption also a point in L, so ` and L have two distinct points in common:
%P and Q. But L and ` are distinct because O is in ` but not in L. This contradicts
%Lemma 1 (page 81) and Theorem 1 is proved.
%
%Corollary. Given a line L and point P not on L, there is a line parallel to L and
%passing through P.
%
%Pick Q on L.  Let O be the midpoint of PQ, and use Theorem 1.  
%
%Theorem 2. Two lines perpendicular to the same line are either identical or parallel

\begin{prob}
Use adjacent angles to prove that vertical angles are equal.    
\end{prob}

\begin{prob}
Now use rotations to prove that vertical angles are equal.
\end{prob}

\begin{prob}
Prove:  Alternate interior angles and corresponding angles of a transversal with respect to a pair of parallel lines are equal.
\end{prob}

\begin{prob}
Prove:  The angle sum of a triangle is 180 degrees.
\end{prob}

\begin{prob}
Prove: If a pair of alternate interior angles or a pair of corresponding angles of a transversal with respect to two lines are equal, then the lines are parallel.
\end{prob}


