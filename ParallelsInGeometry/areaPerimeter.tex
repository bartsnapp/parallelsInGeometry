\newpage

\section{Area and Perimeter}

\begin{prob} You have been asked to put together the dance floor for your sister's wedding.  The dance floor is made up of 24 square tiles that measure one meter on each side. 
\begin{enumerate}
\item Experiment with different rectangles that could be made using all of these tiles, and record your data in a table.  
\item Draw a graph of your data.  Describe patterns in the data, as seen in the table or graph.  
\end{enumerate}
\end{prob}

\begin{prob} Suppose the dance floor is held together by a border made of edge pieces one meter long.  
\begin{enumerate}
\item What determines how many edge pieces are needed: area or perimeter?  Explain. 
\item Make a graph showing the perimeter vs. length for various rectangles with an area of 24 square meters.  
\item Describe the graph.  How do patterns that you observed in the table show up in the graph?  
\item Which design would require the most edge pieces?  Explain.  
\item Which design would require the fewest edge pieces?  Explain.
\end{enumerate}
\end{prob}

\begin{prob}
Suppose you had begun with a different number of floor tiles, such as 30, 21, or 19, or 36.  
\begin{enumerate}
\item In general, describe the rectangle with whole-number dimensions that has the greatest perimeter for a fixed area.  
\item Which rectangle has the least perimeter for a fixed area?
\end{enumerate}
\end{prob}

\begin{prob}  Consider the graphs you drew in the previous problems.  
\begin{enumerate}
\item Can we connect the dots in the graphs?  Explain. 
\item How might we change the context so that the dimensions can be other than whole numbers?   In the new context, how would the previous answers change?
\end{enumerate}
\end{prob}

\begin{prob}
The previous problems were about rectangles with constant area and changing perimeter.  
\begin{enumerate}
\item Make up a problem about rectangles with whole-number dimensions, constant perimeter, and changing area.   
\item Make a table of length, width, perimeter, and area for these rectangles.
\item Draw a graph of width versus length for your rectangles. 
\item Draw a graph of area versus length for your rectangles.  
\item Now modify the context and your graphs to allow dimensions that are not whole numbers.   
\item Which rectangle will have a maximum area? 
\item Which rectangle will have a minimum area? 
\end{enumerate}
\end{prob}

\begin{prob}
So far we have considered rectangles with fixed area and those with fixed perimeter.  What about fixing the width or the length?  Since they behave in much the same way, let's fix the width.   
\begin{enumerate}
\item Make up a problem about rectangles with constant width and changing area and perimeter.   
\item Make a table of length, width, perimeter, and area for these rectangles.
\item Draw a graph of area versus length for your rectangles.  
\item Draw a graph of perimeter versus length for your rectangles. 
\item What kinds of functions do you see?  
\end{enumerate}
\end{prob}

\begin{prob}
Explain how and where you saw the following advanced algebra ideas in the above problems:  
\begin{enumerate}
\item Domain, range and ``limiting cases''
\item Rates of change, maxima, minima, and asymptotic behavior
\item Generalizing from a specific to a generic fixed quantity
\item Equation solving with several variables
\item Distinguishing among various types of functions
\end{enumerate}
\end{prob}
